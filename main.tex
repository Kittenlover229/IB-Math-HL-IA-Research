\documentclass[12pt, titlepage]{article}
\usepackage[skip=5pt plus1pt, indent=20pt]{parskip}
\usepackage{indentfirst}
\usepackage[utf8]{inputenc}
\usepackage[datesep=/,style=ddmmyyyy]{datetime2}
\usepackage{xcolor}
\usepackage[english]{babel}
\newcommand{\TODO}{\begin{center}\color{red}TODO\end{center}}
\newcommand\wordcount{\input{|"texcount -inc -sum -0 -template={SUM} \jobname.tex"}}

\title{Obstacle-avoidant Function Fitting}

\author{Artur Roos}

\date{October 2022}

\begin{document}
\maketitle
\tableofcontents

{
\centering
Document compiled \today, made with \LaTeX.
}

\section{Introduction}
\subsection{Rationale}
Not longer than two days ago I was challenged to a mathematical standoff by a 
good friend of mine, the weapons would be polynomial functions and their 
graphs. Winning this competition was a matter of honor, so I used all of my 
skill and wits to prepare. The fight would be held in \mbox{"Graph Wars"}, a 
small competitive 1v1 game published on the 23rd of February 2022. 

The goal of the game is to design a function that can be plotted from one point
to another without intersecting any circles in shortest time. You are given 
60 seconds to input as many functions as you want, then the turn is passed onto
your opponent.
This is wrapped into a war-like setting where your function is a projectile, 
directed by your soldiers against your enemy's.

It is also usually played without assistance of any external programs like 
plotters or calculators, that's for good reason. Most importantly, game is
considered solved: for any initial state of the field there can be deduced a 
function to guarantee a "hit". 

\subsection{Aim}
My ultimate goal  set out to be simply designing an algorithm, which when given
a list of all circle obstacles, their radii, and the edge points would express
the path to go from to another without colliding with any of the obstacles. 

\section{Investigation}
The task consists of the small independents subtasks: finding the points and 
fitting the function. Task-specific restrictions are mentioned in the 
corresponding section.

\subsection{Finding The Points}
This problem can be boiled down to pathfinding. However, usual pathfinding
techniques operate on discrete spaces like graphs and grids which our plane
is not, we need to convert it to one. Cells of the grid can be in two states:
occupied or empty. The occupied cells are the ones covered by an obstacle. 
Since obstacle circles only occur at integer coordinates and the smallest
radius of a circle is 1, we can safely assume that the smallest size of a grid
cell we might need is 1 respectively. After that we can apply the \textbf{Theta*}.
It is preferable to Dijkstra's, A* and others since it is designed to fit
any-angle pathfinding. Even though our points are aligned on a grid,
the "projectile" is allowed to move in all directions, not just the cardinal ones. 
Only additional piece of information it requires is a line of sight function ($L$).
Let $I(A, B)$ be some function, equal to the amount of intersections of a 
line between points $A$ and $B$  with all the circles of a set $C$ containing 
$\langle x, y, r \rangle$. So $I_i(A, B)$ is the amount of intersections
of a line from $A$ to $B$ with some circle $C_i$. 
Using that we can define $I(A, B) = \sum_{i=1}^{|C|}I_i(A, B)$.

For simplicity let's assume the circle is at $\langle 0, 0 \rangle$ with
some radius $r$. The \textbf{line} of sight between two points $p_1$ and $p_2$
can be expressed as some $y = kx + b'$. 
But first let's find the point ($p_0 = (x_0, y_0)$) on the line closest to the circle.
If we rearrange our equation to be $ax + by + c = 0$ we can for sure say that 
vector $\vec{v^{\star\star}} = (a, b)$ will be perpendicular to the line.

\begin{equation}
    k = \frac{p_{1_y} - p_{2_y}}{p_{1_x} - p_{2_x}}
\end{equation}
\begin{equation}
    b' = p_{1_x}
\end{equation}

Coordiantes of $p_0$ should be proportional to the vector $\vec{v^{\star\star}}$. 
Furthermore, the distance
of $p_0$ from the origin is $d = \frac{|c|}{\sqrt{a^2 + b^2}}$.
Now we may just normalize the vector to the length (distance from origin).

\begin{equation}
    \vec{v^\star}=\left(a * \frac{c}{a^2 + b^2}, b * \frac{c}{a^2 + b^2}\right)
\end{equation}

This vector is inverted, so by multiplying everything by $-1$ we get the actual 
coordinates of the point.

\begin{equation}
    x_0 = -a * \frac{c}{a^2 + b^2}
\end{equation}
\begin{equation}
    y_0 = -b * \frac{c}{a^2 + b^2}
\end{equation}

\TODO

\subsection{Fitting The Function}
Let $A$ be a set of cardinality $n$ where each point is like $\langle x, y \rangle$.
For simplicity we can impose a strict total ordering on the set, and reindex
it in such way that $\forall a_1,a_2 \in A : a_1 < a_2 \Leftrightarrow a_{1_x} < 
a_{2_x}$. In this way the $A_n$ is guaranteed to be the rightmost point, which
will make the calculations a lot easier. Let's also move the coordinate system
so that the point at bottom left $\langle -50, -20 \rangle$ is at the origin.
That will make all the coordinates positive and they are easier to reason about.
We may also ignore all the $x$ values in ranges 
$(-\infty; a_{1_x})$ and $(a_{n_x}; \infty)$, since the function will not be 
evaluated there at any point.

There is an infinite number of ways to make a function that goes through a set of
points, the simplest one is the Lagrange's Interpolation which does fit this 
usecase perfectly. We may follow a specific example and try to generalise it later. 
For example $A = \{\langle 0, 1 \rangle, \langle 6, 9 \rangle, 
\langle 17, 4 \rangle, \langle 19, 22 \rangle\}$.

First we may try to make a function $\phi_i(x)$ such that it equates to 0 at
every point, but $x_i$ it should be. For simplicity we may make it equate to 1, this will
allow for better composability in the future. For example the process for the second point
might look something like this.

First we use a helper function $\hat{\phi}_2(x)$ make it be zero at all the points, but $x_i$.
\begin{equation}
    \hat{\phi}_2(x) = (x - 0)(x - 17)(x - 19)
\end{equation}

Now the value at $x_2$ is non-zero, we can just devide it by $\hat{\phi}_2(x_2)$ 
so it is $1$ at $x_2$.

\begin{equation}
    \hat{\phi}_2(x_2) = (6 - 0)(6 - 17)(6 - 19)
\end{equation}
\begin{equation}
    \phi_2(x) = \frac{(x - 0)(x - 17)(x - 19)}{\hat{\phi}_2(x_2)}
\end{equation}

\begin{equation}
    \phi_2(x) = \frac{(x - 0)(x - 17)(x - 19)}{(6 - 0)(6 - 17)(6 - 19)}
\end{equation}

{
\centering
\begin{tabular}{c|c c c c}
    $x$ & $x_1$ & $x_2$ & $x_3$ & $x_4$ \\
    $\phi_2(x)$ & $0$ & $1$ & $0$ & $0$ \\
\end{tabular}\par
}

To make $\phi_i$ completely representative of the point at $x_i$ we need to 
scale it by $y_i$. Let's define a new helper function $\psi_i(x)$.

\begin{equation}
    \psi_2(x) = y_2 \phi_2(x)
\end{equation}
\begin{equation}
    \psi_2(x) = \frac{(x - x_1)(x - x_3)(x - x_4)}{(x_2 - x_1)(x_2 - x_3)(x_2 - x_4)} * y_2
\end{equation}

Based on this we can express $\psi_i(x)$'s values for any $x_i$.

{
\centering
\begin{tabular}{c|c c c c}
    & $\psi_1(x) $ & $\psi_2(x)$ & $\psi_3(x)$ & $\psi_4(x)$ \\
    \hline
    $x_1$ & $y_1$ & $0$ & $0$ & $0$ \\
    $x_2$ & $0$ & $y_2$ & $0$ & $0$ \\
    $x_3$ & $0$ & $0$ & $y_3$ & $0$ \\
    $x_4$ & $0$ & $0$ & $0$ & $y_4$ \\
\end{tabular}\par
}

Looking at the table of all the values of $\psi_i$ we can deducde that their linear
combination will satisfy our condition of $\psi_i(x_i) = y_i$ and 
$\psi_i(x_j) = 0, j \neq i$, yielding an $|A|$th order polynomial. 
So for the example set $A$ the fit function is 
$f(x) = \psi_1(x) + \psi_2(x) + \psi_3(x) + \psi_4(x)$.

If we generalise this function to any set $A$ it would look something 
like the following.

$$f(x) = \sum_{i=1}^{|A|}\psi_i(x)$$
$$\psi_i(x) = y_i \phi_i(x)$$
\begin{equation}
    \phi_i = \prod_{j=1, j \neq i}^{|A|}\frac{x - x_j}{x_i - x_j}
\end{equation}
\begin{equation}
f(x) = \sum_{i=1}^{|A|} \left(y_i * \prod_{j=1, j \neq i}^{|A|} \frac{x - x_j}{x_i - x_j}\right)
\end{equation}

\section{Limitations \& Reflection}
I used a geometric approach to finding intersections of lines with circles.
The algebraic solution has a higher computational error due to imprecisions 
when using floating-point arithmetic. This doesn't impact the current usecase,
however can be important when dealing with more fine point placement.

The described method of fitting covers a lot of scenarios, but it doesn't yield 
correct results
when the next intended path point $a_{i + 1}$ on the path which goes after 
some $a_i$. In this case the pathfinding algorithm requires a path to turn back,
which is impossible to do. This issue is most visible when reindexing: the order
of points will be changed. Beyond that, this arises only in situations when the 
$f(x)$ \textbf{can't be a well-defined function}, since that requires it to have to
two different values at same $x$. Luckily, those scenarios never occured in a 
real game. Perhaps the game accounts for that and intentionally avoids it.

My method produces good results on the given dataset of densely packed points,
however, the more spread apart the points are the greater the algorithm
overshoots in spaces between two points. That can be a problem since it may 
actually accidentally go through one of the spheres. This effect can be
mitigated by sampling more intermediate points, which in turn increases 
computational complexity. Since the game field is  limited to span 100 units
horizontally and 40 units vertically this is not a big problem. One of the
alternative solutions would be using splines to form B\'{e}zier Curves, 
Hermite Curves Catmul-Rom or even B-Splines, but those are \textbf{barely touched 
on in the book} if at all.

\section{Conclusion}
\TODO

\section{Bibliography}
\appendix
\TODO

\end{document}
