\documentclass[12pt, titlepage]{article}
\usepackage[utf8]{inputenc}
\usepackage[english]{babel}

\title{}

\author{Artur Roos}

\date{October 2022}

\begin{document}
\maketitle
\tableofcontents

\section{Introduction}
\subsection{Rationale}
Not longer than two days ago I was challenged to a mathematical standoff by a 
good friend of mine, the battlefield would be polynomial functions and their 
graphs. Winning this competition was a matter of honor, so I used all of my 
skill and wits to prepare. The fight would be held in \mbox{"Graph Wars"}, a 
small competitive 1v1 game published on the 23rd of February 2022. 

The goal of the game is to design a function that can be plotted from one point
to another without intersecting any circles faster than your opponent. This
is wrapped into a war-like setting where your function is a projectile, directed
by your soldiers against your common foes.

It is also usually played without assistance of any external programs like 
graph plotters or calculators, that's for good reason. Most importantly, game is
considered solved: for any initial state of the field there can be deduced a 
function to guarantee a "hit". 

\subsection{Aim}
I would want th algorithm to be fully automate, so my ultimate goal 
set out to be simply designing an algorithm, which when given a list of all
circle obstacles, their radii, and the edge points would express the path to
go from to another without colliding as a polynomial.

\section{Investigation}
The task consists of the small independents subtasks: finding the points and 
fitting the function. 

\subsection{Finding The Points}
This problem can be boiled down to pathfinding. However, usual pathfinding
tehcniques operate on discrete spaces like graphs and grids which our spaces
is difficult to represent.

\subsection{Fitting The Function}
Let $A$ be a set of length $n$ where each point is like $\langle x, y \rangle$.
For simplicity we can impose a strict total ordering on the set, and reindex
it in such way that $\forall a_1,a_2 \in A (a_1 < a_2 \Leftrightarrow a_{1_x} < 
a_{2_x})$. In this way the $L_n$ is guaranteed to be the rightmost point, which
will make the mathematics a lot easier. Let's also move the coordinate system
so that the point at bottom left $\langle -50, -20 \rangle$ is at the origin.
That will make all the coordinates positive and they are easier to reason about.
Simple transation $g(p) = \langle p_x + 50, p_y + 20 \rangle, p \in A$ does the
trick. To reverse this transformation later we might just use $g^{-1}$.

There is an infinite number of ways to make a function that goes through a 
point, but since we have a generated path as a straight line, 
we should consider going in a straight line too. 
The equation of a line $kx + b$ is useful, but it only allows us to
connect two points, but we have many more. If we express our function as 
piecewise $u(x)$ we may notice that it resebles the way an absolute value function 
behaves when expanded at different values of $x$. That function also happens
to be our target function. 

\begin{equation}
    \left\{\begin{array}{@{}l@{}}
        u(x) = k_1x + b_1, x \in [a_{1_x}, a_{2_x}] {}\\
        u(x) = k_2x + b_2, x \in [a_{2_x}, a_{3_x}] {}\\
        \cdots\\
        u(x) = k_nx + b_n, x \in [a_{n-1_x}, a_{n_x}] {}\\
    \end{array}\right.\,
    a \in A
\end{equation}

\section{Limitation}
This method covers a lot of scenarios, but it doesn't yield correct results
when the next intended point on the path $a_2$ which goes after some $a_1$
satisfied $a_{2_x} < a_{1_x}$. Because of the reindexing this changes the order
of points. Beyond that, this arises only in situations when the $f$ is not
a well-defined function, since that requires it to have to two different values
at same $x$. Luckily, those scenarios never occured in a real game. Perhaps 
the game accounts for that and intentionally avoids it.

\section{Reflection}

\section{Conclusion}

\end{document}
