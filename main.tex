\documentclass[12pt, titlepage]{article}
\usepackage[utf8]{inputenc}
\usepackage{xcolor}
\usepackage[english]{babel}
\newcommand{\TODO}{\begin{center}\color{red}TODO\end{center}}

\title{}

\author{Artur Roos}

\date{October 2022}

\begin{document}
\maketitle
\tableofcontents

\section{Introduction}
\subsection{Rationale}
Not longer than two days ago I was challenged to a mathematical standoff by a 
good friend of mine, the battlefield would be polynomial functions and their 
graphs. Winning this competition was a matter of honor, so I used all of my 
skill and wits to prepare. The fight would be held in \mbox{"Graph Wars"}, a 
small competitive 1v1 game published on the 23rd of February 2022. 

The goal of the game is to design a function that can be plotted from one point
to another without intersecting any circles faster than your opponent. This
is wrapped into a war-like setting where your function is a projectile, directed
by your soldiers against your common foes.

It is also usually played without assistance of any external programs like 
graph plotters or calculators, that's for good reason. Most importantly, game is
considered solved: for any initial state of the field there can be deduced a 
function to guarantee a "hit". 

\subsection{Aim}
I would want th algorithm to be fully automate, so my ultimate goal 
set out to be simply designing an algorithm, which when given a list of all
circle obstacles, their radii, and the edge points would express the path to
go from to another without colliding with any of the collided obstacles.

\section{Investigation}
The task consists of the small independents subtasks: finding the points and 
fitting the function. 

\subsection{Finding The Points}
This problem can be boiled down to pathfinding. However, usual pathfinding
techniques operate on discrete spaces like graphs and grids which our plane
is not, we need to convert it to one. Cells of the grid can be in two states:
occupied or empty. The occupied cells are the ones covered by an obstacle. 
Since obstacle circles only occur at integer coordinates and the smallest
radius of a circle is 1, we can safely assume that the smallest size of a grid
cell we might need is a half. After that we can apply the A* pathfinding
algorithm to yield us a set of points $A$, those are the points that the
function should go through. 

\TODO{}

\subsection{Fitting The Function}
Let $A$ be a set of length $n$ where each point is like $\langle x, y \rangle$.
For simplicity we can impose a strict total ordering on the set, and reindex
it in such way that $\forall a_1,a_2 \in A (a_1 < a_2 \Leftrightarrow a_{1_x} < 
a_{2_x})$. In this way the $L_n$ is guaranteed to be the rightmost point, which
will make the mathematics a lot easier. Let's also move the coordinate system
so that the point at bottom left $\langle -50, -20 \rangle$ is at the origin.
That will make all the coordinates positive and they are easier to reason about.
Simple transation $g(p) = \langle p_x + 50, p_y + 20 \rangle, p \in A$ does the
trick. To reverse this transformation later we might just use $g^{-1}$. We may
also ignore all the $x$ values in ranges 
$(-\infty; a_{1_x})$ and $(a_{n_x}; \infty)$, since the function will not be 
evaluated there at any point.

There is an infinite number of ways to make a function that goes through a set of
points, the simplest one is the Lagrange Interpolation which does fit this 
usecase perfectly.

Sadly, the game does not accept piecewise functions so we have to design a
way to express it as a, for example, polynomial.
We may follow a specific example and try to generalise it later. 
Let $A' = \{\langle 0, 1 \rangle, \langle 6, 9 \rangle, 
\langle 17, 4 \rangle, \langle 19, 22 \rangle\}$.

First we may try to make a function $\phi_i(x)$ such that it equates to 0 at
every point, but $x_i$. For simplicity we may make it equate to 1, this will
allow for better composability in the future. For the second point the process
might look something like this:

First make it be zero at all the points, but $x_i$.
\begin{equation}
    \hat{\phi}_2(x) = (x - 0)(x - 17)(x - 19)
\end{equation}

Now the value at $x_2$ is non-zero, we can just devide it by $\hat{\phi}_2(x_2)$.

\begin{equation}
    \hat{\phi}_2(x_2) = (6 - 0)(6 - 17)(6 - 19)
\end{equation}
\begin{equation}
    \phi_2(x) = \frac{(x - 0)(x - 17)(x - 19)}{\hat{\phi}_2(x_2)}
\end{equation}

\begin{equation}
    \phi_2(x) = \frac{(x - 0)(x - 17)(x - 19)}{(6 - 0)(6 - 17)(6 - 19)}
\end{equation}
$$\phi_2(x_1) = 0, \phi_2(x_2) = 1\\,\phi_2(x_3) = 0\\, \phi_2(x_4) = 0\\$$

To make $\phi_i$ completely representative of the point at $x_i$ we need to 
scale it by $y_i$. Let's define a new helper function $\psi_i(x)$.

\begin{equation}
    \psi_2(x) = \frac{(x - x_1)(x - x_3)(x - x_4)}{(x_2 - x_1)(x_2 - x_3)(x_2 - x_4)} * y_2
\end{equation}
\begin{equation}
    \psi_2(x) = y_2 \phi_i(x)
\end{equation}
$$\psi_2(x_1) = 0, \psi_2(x_2) = 9\\,\psi_2(x_3) = 0\\, \psi_2(x_4) = 0\\$$

Looking at the table of all the values of $\psi_i$ we can deducde that their linear
combination will satisfy our condition of $\psi_i(x_i) = y_i$ and 
$l_i(x_j) = 0, j \neq i$. So for set $A'$ the fit function is 
$$f(x) = \psi_1(x) + \psi_2(x) + \psi_3(x) + \psi_4(x)$$

If we generalise this function to for any set $A$ it would look something 
like the following.

$$f(x) = \sum_{i=1}^{|A|}\psi_i(x)$$
$$\psi_i(x) = y_i \phi_i(x)$$
\begin{equation}
    \phi_i = \prod_{j=0, j \neq i}^{|A|}\frac{x - x_j}{x_i - x_j}
\end{equation}
\begin{equation}
f(x) = \sum_{i=1}^{|A|} \prod_{j=0, j \neq i}^{|A|}y_i * \frac{x - x_j}{x_i - x_j}
\end{equation}

\section{Limitations}
This method covers a lot of scenarios, but it doesn't yield correct results
when the next intended point on the path $a_2$ which goes after some $a_1$
satisfied $a_{2_x} < a_{1_x}$. Because of the reindexing this changes the order
of points. Beyond that, this arises only in situations when the $f$ is not
a well-defined function, since that requires it to have to two different values
at same $x$. Luckily, those scenarios never occured in a real game. Perhaps 
the game accounts for that and intentionally avoids it.

\section{Reflection}
\TODO

\section{Conclusion}
\TODO

\end{document}
