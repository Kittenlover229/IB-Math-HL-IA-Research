\documentclass[12pt, titlepage]{article}
\usepackage[utf8]{inputenc}
\usepackage[english]{babel}

\title{}

\author{Artur Roos}

\date{October 2022}

\begin{document}
\maketitle

\section{Introduction}
\subsection{Rationale}
Not longer than two days ago I was challenged to a mathematical standoff by a 
good friend of mine, the battlefield would be polynomial functions and their 
graphs. Winning this competition was a matter of honor, so I used all of my 
skill and wits to prepare. The fight would be held in \mbox{"Graph Wars"}, a 
small competitive 1v1 game published on the 23rd of February 2022. 

The goal of the game is to design a function that can be plotted from one point
to another without intersecting any circles faster than your opponent. This
is wrapped into a war-like setting where your function is a projectile, directed
by your soldiers against your common foes.

It is also usually played without assistance of any external programs like 
graph plotters or calculators, that's for good reason. The game is considered
solved: for any initial state of the field there can be deduced a function to
guarantee a "hit". 

\subsection{Aim}
I would want th algorithm to be fully automate, so my ultimate goal 
set out to be simply designing an algorithm, which when given a list of all
circle obstacles, their radii, and the edge points would express the path to
go from to another without colliding as a polynomial.

\section{Investigation}
The task consists of the small independents subtasks: finding the points and 
fitting the function. 

\subsection{Finding The Points}
\subsection{Fitting The Function}

\section{Reflection}
\section{Conclusion}

\end{document}
